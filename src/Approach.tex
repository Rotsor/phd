\chapter{Approach}

\subsection{Abstract of Balsa Resynthesis}

The paper proposes a modification of the standard design workflow
that is used in Balsa design automation system. The controllers obtained
by syntax-directed mapping used in Balsa usually suffer from performance,
area and power overheads because the predesigned set of components
is required to implement the declared protocols fully and correctly
in order to be reusable in all possible circuit configurations, which
results in redundancy. This redundancy can be eliminated by replacing
the manually designed gate-level implementations of the high level
components with the corresponding STG specifications. The STGs of
individual components that form the system are then composed together
to produce the final system STG that is used to synthesise an optimal
implementation of the control circuit. The process is automated as
a plug-in for Workcraft framework.

\subsection{Introduction of Balsa Resynthesis\label{sec:Balsa-Introduction}}

The main obstacle for the wider spread of asynchronous systems remains
to be the inherent complexity of their design. Several solutions are
accepted by the industry that ease the design process through abstraction
of predesigned asynchronous circuit parts as standardised high level
components. A designer is able to use these components as ``building
blocks'', and then obtain the final gate-level design through an
automated mapping process. Furthermore, some of the well-known asynchronous
design automation packages, such as Tangram~\cite{951597}, and Balsa~\cite{balsa},
define a high-level programming-like language that is used to describe
systems. The language constructs are then directly translated into
a network of \emph{handshake components }-- blocks with predefined
functionality that use \emph{handshakes} to interface with other components,
which are in turn mapped into a gate netlist. 

Although this method greatly enhances the designer's productivity,
it has several important drawbacks, of which the control-path overhead
is the most decisive. The controllers obtained by syntax-directed
mapping are usually far from optimal, because the predesigned components
are required to implement their declared protocols fully and correctly
in order to be reusable in all possible circuit configurations. However,
it is often the case that a significant part of their functionality
becomes redundant due to the peculiarities of the specific configuration,
e.g. in many cases full handshaking between the components can be
avoided.

This redundancy can be eliminated by replacing the manually designed
gate-level implementation of the high level components with an equivalent
STG~(signal transition graph)~\cite{Yakovlev_1998_cs} specification.
The individual component STGs are then composed together to form a
complete system STG~\cite{785214}, which is optimised using \noun{petrify~}\cite{cortadella_petrify}\noun{.}
An optimal gate-level implementation can then be automatically produced
from the STG using tools such as \noun{petrify}~\noun{\cite{cortadella_petrify},
SIS}~\cite{Sentovich:M92/41}\noun{ }and\noun{ MPSat}~\cite{Khomenko_2004_MPSAT}\noun{.}
Automatic synthesis becomes problematic when the size of the STG becomes
large: modern synthesis tools can handle STGs of no more than 100
signals. The impact of this problem can be lessened by including STG
decomposition tools~\cite{DesiJ} into the workflow, that would break
the large optimised STG down into several smaller STGs that are synthesisable
in reasonable time. Alternatively, the decomposition step can be carried
out on the level of the handshake circuits, dividing the circuit into
smaller blocks of components.

This paper proposes an automated method to include the aforementioned
modification of the standard design workflow that is used in Balsa
design automation system~\cite{balsa} using \noun{Workcraft~\cite{DBLP:conf/apn/PoliakovKY09}
}framework.
