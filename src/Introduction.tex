
\chapter{Introduction}

We explore the available formalisms for design of asynchronous digital
circuits and compare them with the principle of compositionality in
mind. We propose improvements to the existing techniques and
propose a new formalism, Parametrised Graph (PG) theory, which is a
generalisation of the existing CPOG formalism. We furhter study 
the theory of PG Algebra and present mechanised formal proofs of certain properties.


\section{Motivation}

With constantly growing transistor counts and consumer demands, the complexity of digital circuits must grow too.

With the rising complexity of digital circuits, it becomes increasingly important to reuse parts of existing designs.

The reuse of the components must be facilitated by the design language.

One of the difficulties in designing modern hardware systems is the
necessity to comprehend and to deal with a very large number of system
configurations, operational modes, and behavioural scenarios. 

\subsection{Compositionality}

The key property facilitating component reuse is compositionality.

Compositionality is a property of a given language where the meaning of a language construct is determined by the meanings of its parts.


\section{Contributions}

The main contributions of the thesis are as follows:

\begin{itemize}
\item
\textbf{Improved parallel composition:} a novel method for composition of models specified with labelled Petri Nets.

\item
\textbf{Balsa circuit synthesis:} application of labelled Petri nets and parallel composition to synthesis of Balsa handshake circuits.

\item
\textbf{CPOG Synthesis:} a technique for synthesis of processor instruction decoder using instruction sets specified with Conditional Partial Order Graphs.

\item
\textbf{PG theory:} formal specification of theory of parameterised graphs with CPOGs as an example of its algebra.

\item
\textbf{CAD tool support:} automation for design of CPOGs and Balsa circuits using Workcraft framework.

\end{itemize}

\section{Structure}

The rest of the thesis is organised as follows:

Chapter~\ref{chap:Background} covers the basics of handshake circuits, signal transition graphs and conditional partial order graphs.

Chapter~\ref{chap:Approach} overviews the existing composition methods and outlines the drawbacks using an intuitive GCD benchmark as an example.

Chapter~\ref{chap:ParComp} describes the proposed improved parallel composition algorithm.  .

Chapter~\ref{chap:PGAlgebra} introduces Parametrised Graph (PG) theory, defining and studying an algebraic structure that generalises Conditional Partial Order Graph formalism.

Chapter~\ref{chap:PGEncoding} describes a technique for optimal encoding of processor instruction sets defined using CPOG formalism.

Chapter~\ref{chap:Conclusion} summarises the achieved results and proposes ideas for future research.

