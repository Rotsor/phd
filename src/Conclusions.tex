\chapter{Conclusions\label{chap:Conclusion}}

In this thesis we have presented a number of solutions to the problem
of compositional design of large scale digital circuits. These
solutions help to synthesise efficient asynchronous control circuits,
transform and analyse compositional structures suitable for subsequent
generation of hardware control logic and compute an efficient encoding
of micro-controller instruction set.



\section{Improved parallel composition}
In Chapter \ref{chap:ParComp} we discussed an improved algorithm for
computing the parallel composition of STGs or labelled Petri nets. The
algorithm makes use of the FCI assumption to produce nets with fewer
implicit
places. This aids in the synthesis of  subsequent structural
algorithms like
dummy contraction. It uses only simple structural checks and
thus is very efficient even for large compositions, so the
improvement comes at negligible cost.

The algorithm was implemented in the \pcomp tool and evaluated
on a set of scalable benchmarks. The experiments proved its
efficiency, which increases even more when the components are
pre-processed to remove dummies and ensure injective labelling
(this is usually cheap, as the components are small; moreover,
if the components come from a standard library of component
types, this step can be completely eliminated).

Another important advantage is that the improved algorithm
places almost no additional effort on the user: the only
requirement is to pass an additional command-line option to
\pcomp so that it can assume the FCI property and apply the
proposed optimisation.


\section{Parametrised Graphs theory}

We introduced a new formalism called Parametrised Graphs and the
corresponding algebra. The formalism allows to manage a large number
of system configurations and execution modes, exploit similarities
between them to simplify the specification, and to work with groups
of configurations and modes rather than with individual ones. The
modes and groups of modes can be managed in a compositional way, and
the specifications can be manipulated (transformed and/or optimised)
algebraically in a fully formal and natural way.

We develop two variants of the algebra of parametrised graphs,
corresponding
to the two natural graph equivalences: graph isomorphism and isomorphism
of transitive closures. Both cases are specified axiomatically, and
the soundness, minimality and completeness of the resulting sets of
axioms are formally proved. Moreover, the canonical forms of algebraic
terms are developed in each case.

We have formalised the definitions of algebra of parametrised graph in
Agda, and developed the machine-checked proofs of several properties
of that algebra.

The formula representation data structure was designed together with
the custom structural equivalence relation on formula representations
for convenient formula manipulations. The equivalence relation has
been proved equivalent to the one defined using formula semantics,
thus showing its adequacy.

The normal form representation data structure for PG formulae was
designed with its semantics defined as translation to the
corresponding general formula representations. The algorithm of
finding the normal form of general formulae was developed and was
proved to be correct.

The usefulness of the developed formalism has been demonstrated on
two case studies, a phase encoding controller and a processor
micro-controller.
Both have a large number of execution scenarios, and the developed
formalism allows to capture them algebraically, by composing individual
scenarios and groups of scenarios. The possibility of algebraical
manipulation was essential to obtain the optimised final specification
in each case.

The developed formalism is also convenient for implementation in a
tool, as manipulating algebraic terms is much easier than general
graph manipulation; in particular, the theory of term rewriting can
be naturally applied to derive the canonical forms.


\section{Processor instruction set encoding}

The Chapter~\ref{chap:PGEncoding} presented the PG model based
methodology for micro architecture
design and studied its application for specification and synthesis
of a central processor micro-controller. The key contribution is
the method for synthesis of optimal instruction op-codes; the
corresponding
optimisation problem is formulated in terms of PGs and reduced to
the well-known Boolean satisfiability problem. The method is implemented
in a software tool and can be used within the conventional micro
architecture
design flow.

The studied processor example is purely academic. Nonetheless, it captures
many important features of real processors. It has been demonstrated
that the PG model is capable of modelling concurrency between different
subsystems and handling multiple choice during instruction execution.
Future work focuses on specification and synthesis of a real processor
and optimisation of the encoding algorithms.

Both have a large number of execution scenarios, and the developed
formalism allows to capture them algebraically, by composing individual
scenarios and groups of scenarios. The possibility of algebraical
manipulation was essential to obtain the optimised final specification
in each case.

The developed formalism is also convenient for implementation in a
tool, as manipulating algebraic terms is much easier than general
graph manipulation; in particular, the theory of term rewriting can
be naturally applied to derive the canonical forms.


%%%%

\section{Future work}
The improved parallel composition from Chapter \ref{chap:ParComp} can
be generalised by weakening the test used in implicit place
elimination and thus taking into the consideration the non-trivial
relationships between inputs and outputs. At the moment, we only
remove placed directly connected to input and output transitions; a
stronger version of Proposition \ref{pr-main} would reason not at
the level of individual places, immediately connected to input/output
transitions, but rather whole paths going from an output transition to
an input transition.

For Parametrised Graphs, we plan to automate the algebraic
manipulation of PGs, and implement automatic synthesis of PGs into
digital circuits. For
the latter, much of the code developed for the precursor formalism
of Conditional Partial Order Graphs (CPOGs) can be re-used. One of
the important problems that needs to be automated is that of
simplification
of (T)PG expressions, in the sense of deriving an equivalent expression
with the minimum possible number of operators. Our preliminary research
suggests that this problem is strongly related to modular decomposition
of graphs~\cite{2005_McConnell_modular}.

We also plan to formalise the proof of CPOG being a model of PG
Algebra and modification of the normalisation algorithm from
Chapter~\ref{chap:PGAlgebra}  computing the canonical form (where each
graph node is mentioned no more than once) instead of just a normal
form. Canonical form is much more useful because its size is at most
quadratic while the size of a normal form is exponential in general.
The canonical form is also a way to quickly compare graphs and solve
equational relations.
