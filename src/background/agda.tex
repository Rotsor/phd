%include lhs2TeX.fmt
%include polycode.fmt
\usepackage{amstext}
\usepackage{amssymb}
\usepackage{stmaryrd}

\DeclareMathAlphabet{\mathkw}{OT1}{cmss}{bx}{n}

\newcommand{\redFG}[1]{\textcolor[rgb]{0.6,0,0}{#1}}
\newcommand{\greenFG}[1]{\textcolor[rgb]{0,0.4,0}{#1}}
\newcommand{\blueFG}[1]{\textcolor[rgb]{0,0,0.8}{#1}}
\newcommand{\orangeFG}[1]{\textcolor[rgb]{0.8,0.4,0}{#1}}
\newcommand{\purpleFG}[1]{\textcolor[rgb]{0.4,0,0.4}{#1}}
\newcommand{\yellowFG}[1]{\textcolor{yellow}{#1}}
\newcommand{\brownFG}[1]{\textcolor[rgb]{0.5,0.2,0.2}{#1}}
\newcommand{\blackFG}[1]{\textcolor[rgb]{0,0,0}{#1}}
\newcommand{\whiteFG}[1]{\textcolor[rgb]{1,1,1}{#1}}
\newcommand{\yellowBG}[1]{\colorbox[rgb]{1,1,0.2}{#1}}
\newcommand{\brownBG}[1]{\colorbox[rgb]{1.0,0.7,0.4}{#1}}

\newcommand{\ColourStuff}{
  \newcommand{\red}{\redFG}
  \newcommand{\green}{\greenFG}
  \newcommand{\blue}{\blueFG}
  \newcommand{\orange}{\orangeFG}
  \newcommand{\purple}{\purpleFG}
  \newcommand{\yellow}{\yellowFG}
  \newcommand{\brown}{\brownFG}
  \newcommand{\black}{\blackFG}
  \newcommand{\white}{\whiteFG}
}

\newcommand{\MonochromeStuff}{
  \newcommand{\red}{\blackFG}
  \newcommand{\green}{\blackFG}
  \newcommand{\blue}{\blackFG}
  \newcommand{\orange}{\blackFG}
  \newcommand{\purple}{\blackFG}
  \newcommand{\yellow}{\blackFG}
  \newcommand{\brown}{\blackFG}
  \newcommand{\black}{\blackFG}
  \newcommand{\white}{\blackFG}
}

% \ColourStuff
\MonochromeStuff

\newcommand{\K}[1]{\yellow{\mathsf{#1}}}
\newcommand{\Q}[1]{\green{\mathsf{#1}}}
\newcommand{\D}[1]{\blue{\mathsf{#1}}}
\newcommand{\C}[1]{\red{\mathsf{#1}}}
\newcommand{\F}[1]{\green{\mathsf{#1}}}
\newcommand{\V}[1]{\purple{\mathit{#1}}}

\newcommand{\dfeq}{\overset{\mathrm{df}}{=}}

%%%%%%% BREAK HERE %%%%%%%

\section{Agda\label{sec:Agda-background}}

Agda~\cite{norell:thesis} is a system serving both as a programming language and as a proof assistant simultaneously.
When viewed as a programming language, it is a purely functional language with its syntax largely inspired by 
Haskell. Its main distinguishing features are totality (the fact that every function is defined 
on every possible input) and dependent typing system that allows for types to depend on the values.

\subsection{Function definitions and algebraic data types}

A very common language construct in Agda is a function definition. Function definition must consist of type signature followed by its defining equations. As a simple example, consider this Boolean exclusive OR function:

\begin{code}
xor : Bool → Bool → Bool
xor x y = x ∧ (¬ y) ∨ y ∧ (¬ x)
\end{code}

Here we define a function called |xor| of type |Bool → Bool → Bool| with two arguments |x| and |y| defined in terms of |_∧_|, |_∨_| and |¬_|.

Functions can have more than one defining equation when each equation defines the function for a particular shape, or \emph{pattern} of arguments. This is called \emph{pattern matching}.
For example, the Boolean negation function can be defined the following way:

\begin{code}
¬_ : Bool → Bool
¬ false = true
¬ true = false
\end{code}

A more interesting example would be a Boolean AND:

\begin{code}
_∧_ : Bool → Bool → Bool
true ∧ true = true
_ ∧ _ = false
\end{code}

This example demonstrates an additional feature of pattern matching: equations are ordered and the earlier ones take precedence.

In the above we used the |Bool| data type to represent Boolean values. This data type is not a built-in language construct of Agda, but can be defined using the |data| keyword:

\begin{code}
data Bool : Set where
  true : Bool
  false : Bool
\end{code}

Here definition gives the name for the type and lists \emph{constructors} of its values: |true| and |false|. In this case constructors have no arguments, thus corresponding to individual values, but in general a single constructor can correspond to multiple values, as will be shown later.

\subsection{Inductive types and recursion}

We often want to reason about data types with infinite number of values, such as natural numbers.
To represent them we use inductive type definitions:

\begin{code}
data ℕ : Set where
  zero : ℕ
  suc : ℕ → ℕ
\end{code}

Here we define natural numbers as something that has two forms: it is either a |zero|, or a successor |suc x| where |x| is another natural number. These two constructors allow us to construct an arbitrary number of values of type |ℕ| by successive application of |suc| to |zero|: |zero| corresponds to 0, |suc zero| corresponds to 1, |suc (suc zero)| to 2, etc.

To manipulate the values of inductive data types we use recursive functions:

\begin{code}
_+_ : ℕ → ℕ → ℕ
zero + y = y
suc x + y = suc (x + y)
\end{code}

Here we define natural number addition recursively by considering the cases for the first argument: the base case of $0 + y$ must evaluate to $y$ and for the recursive case $(1 + x) + y$ must evaluate to $1 + (x + y)$. The Agda compiler checks that the arguments to the function become smaller on recursive calls, thus ensuring that only well-behaved (terminating) definitions are admitted.

\subsection{Indexed types and propositions}

There is a close correspondence between types and
logic, a phenomenon known as Curry-Howard correspondence. Specifically, types can be thought of as propositions where the type is inhabited if and only 
if the corresponding proposition holds true. Similarly, well-typed terms can be thought of as proofs 
of the corresponding propositions. Consequently, the type-checker can be used as a proof checker.

With the language features described so far we can construct types |⊤| and |⊥| corresponding to true (logical tautology) and false (logical contradiction). These are not to be confused with |true| and |false| values of type |Bool| that can not be used as types.

\begin{code}
data ⊤ : Set where
  tt : ⊤
data ⊥ : Set where
\end{code}

Here the type |⊤| has a constructor |tt|, which makes it inhabited, thus corresponding to the logic value of true.
The type |⊥| has no constructors, which makes it uninhabited, thus corresponding to false.

To construct more complex propositions Agda provides parameterised types, dependent function types and indexed inductive data families.

Type parameters are a basic way to allow for generic data types or logic operators. Consider the following example:

\begin{code}
data Both (A : Set) (B : Set) : Set where
  both : A → B → Both A B

data Either (A : Set) (B : Set) : Set where
  left : A → Either A B
  right : B → Either A B
\end{code}

Here, |Both A B| can be thought of as a type of tuples of the form |both x y| with |x : A| and |y : B|.
At the same time, for propositions |P| and |Q|, |Both P Q| can be thought of as their conjunction so that the proof |both p q| 
can be constructed if and only if both |p| and |q| (proofs of |P| and |Q|) can be constructed.

Similarly, |Either| is playing a dual role of taking the disjoint union of its type parameters
and the logical disjunction operator. Type |Either P Q| is inhabited if and only if types |P| or |Q| or both are inhabited.

Another way to define parameterised types is to compute them as a result of a function.

Consider some examples:

\begin{code}
Not : Set → Set
Not P = P → ⊥

IsTrue : Bool → Set
IsTrue true = ⊤
IsTrue false = ⊥

\end{code}

|Not| takes a type |P| and computes another type |Not P|, 
which is inhabited if and only if contradiction is derivable from |P|.
It is useful to think of |Not P| as being inhabited if and only if |P| is not.

|IsTrue x| is a type that is inhabited if and only if a Boolean value |x| happens to equal |true|.

Dependent function types have the form |(x : X) → Y| where |x| can be free in |Y|. 
This lets the type of the function result to depend on the value passed in as the argument.
In the case when |Y| is a logical proposition, the dependent type can be thought of as universal quantification over |x|.
Indeed, let us construct a value that can serve as a proof that for 
any boolean value |x| one of |x| and |¬ x| must be |true|:

\begin{code}
lemma₁ : (x : Bool) → Either (IsTrue x) (IsTrue (¬ x))
lemma₁ true = left tt
lemma₁ false = right tt
\end{code}

Finally, indexed inductive type families give you more flexibility by having the constructor choose the values for type parameter
instead of having to construct a type for a given parameter value:

\begin{code}
data IsEven : ℕ → Set where
  zero : IsEven 0
  suc : (n : ℕ) → IsEven n → IsEven (suc (suc n))
\end{code}

Here |IsEven n| is inhabited if and only if |n| is an even number.

With dependent functions it is often useful to omit some of the arguments because their values are uniquely 
determined by the types of the arguments that follow. To be able to do that you can mark the corresponding
parameter as implicit by putting it in curly braces:

\begin{code}
  lemma₂ : {x : ℕ} → IsEven x → Not (IsEven (suc x))
  lemma₂ zero () 
  lemma₂ (suc e) (suc z) = lemma₂ e z

  3-not-even : Not (IsEven (suc (suc (suc zero))))
  3-not-even = lemma₂ (suc zero)
\end{code}

Here the special syntax of |()| is used to indicate that we are doing a pattern matching on a given parameter with no cases to choose from.
This situation allows you to complete the definition without having to give a right-hand side.

Of special importance is the equality type, |_≡_ : {A : Set} → (x : A) → (y : A) → Set|. We have |x ≡ y| inhabited if and only if |x| and |y|
are the same value. It is useful for equational reasoning that is often more natural than inductive proofs.

% 
% The central syntactic elements of the language are \emph{terms}.
% A term can take one of the following forms:
% 
% \begin{itemize}
% \item
% An name bound earlier, such as $x$.
% \item
% Lambda abstraction, such as $\lambda \mathbf{x} \rightarrow \mathbf{E}$ where $\mathbf{x}$ is a newly-bound name and $\mathbf{E}$ is a term having $x$ in scope.
% \item
% Function application, such as $\mathbf{f} \mathbf{x}$ where both $\mathbf{f}$ and 
% $\mathbf{x}$ are terms.
% \item
% The built-in term for the type of all types, namely $Set$.
% \item
% % % % % % % Dependent function type: $(x : \bathbf{A}) \rightarrow \bathbf{B}$ with both $\bathbf{A}$ and $\bathbf{B}$ being terms, with $\bathbf{B}$ possibly containing a free variable of type $\bathbf{A}$;
% \end{itemize}
