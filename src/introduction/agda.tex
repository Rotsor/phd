\section{Machine-assisted formalisation of parametrised graphs}


Since parametrised graphs are likely to be applied in a safety-critical toolchain, it is imperative to attain a degree of confidence in its properties. Equally important is to convince prospective users that the techniques is sound and lives up to its promises. To fullfil this goal, it was decided to construct in a strict and controlled manner a complete formalisation of the PG formalism. The Agda system \cite{agda} was chosen for its expressive notation language and extensive support for machine-checked formal inference. Agda has enjoyed a notable success as the basis for the formalisation (case studies, success stories)

Says that Agda is LCF, what is LCF. What are the altrenatives: Isaballe/HOL, HOL-Light, ACL2, Coq, PVS, Nqthm. What the have achieved (the colouring problem, pentium div, ...)

compare to Maude (2 lines)

a small agda example: list 

explain why an alegbraic specification is better suited than a model-based (VDM, Z, B) or process based (CSP, CCS).


The new Parametrised Graphs  introducing the following features:
\begin{itemize}
\item{it lifts the assumption of graph acyclicity, allowing general graphs instead of partial orders;}
\item{it adds algebraic operations for combining existing specifications, thus achieving compositionality;}
\item{it axiomatically defines the equivalence relation on specifications, allowing for equivalence preserving transformations.}
\end{itemize}

