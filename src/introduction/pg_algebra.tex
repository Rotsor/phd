\section{PG Algebra}

We continue the work started in~\cite{2010_mokhov_ieee}
where a formal model, called Conditional Partial Order Graphs (CPOGs),
was introduced. Using CPOGs as a foundation allowed us to represent individual system configurations
and operational modes as annotated graphs and to efficiently overlay them by exploiting
their similarities. However, the CPOG formalism lacks the compositionality
and the ability to compare and transform specifications in a rigorous
manner~\cite{pg_algebra}. In particular, CPOGs always represent a specification as
a `flat' structure, similar to the canonical form defined in Section~\ref{sec:Parametrised-Graphs},
hence a hierarchical representation of a system as a composition of
its components is not possible. We extend this formalism in several
ways:

\begin{itemize}
\item We transition from the graphs representing partial orders to general graphs.
Nevertheless, if partial orders are the most natural way to represent
a certain aspect of system, this still can be handled.
\item The new formalism is fully compositional.
\item We describe the equivalence relation between the specifications as
a set of axioms, obtaining an algebra of parametrised graphs. This set of axioms is proved
to be sound, minimal and complete~\cite{pg_algebra}.
\item The developed formalism makes possible to manipulate specifications as
algebraic expressions using the rules of the algebra. This
can be viewed as adding a syntactic level to the semantic representation
of specifications, and is reminiscent of the relationship between digital
circuits and Boolean algebra.
\end{itemize}
We demonstrate the usefulness of the developed formalism on the basis of two case
studies. The first one is concerned with development of a phase encoding
controller that represents information by the order of arrival of
signals on $n$ wires. As there are $n!$ possible arrival orders,
it is a challenge to specify the set of corresponding behavioural
scenarios in a compact way. The proposed formalism not only allows us
to solve this problem but also does it in a compositional manner. The final specification is obtained through the composition of fixed-size fragments
describing the behaviours of a pair of wires (the latter is impossible
with the CPOG formalism).


The second case study is concerned with designing a microcontroller
for a simple processor. The processor can execute several classes
of instructions and each class is characterised by a specific execution
scenario of the operational units of the processor. In turn, the scenarios
of conditional instructions have to be composed of sub-scenarios corresponding
to the current value of the appropriate ALU flag. The overall specification
of the microcontroller is then obtained algebraically by composing
scenarios of each class of instructions.



